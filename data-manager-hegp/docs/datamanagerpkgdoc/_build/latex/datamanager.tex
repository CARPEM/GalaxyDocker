% Generated by Sphinx.
\def\sphinxdocclass{report}
\newif\ifsphinxKeepOldNames \sphinxKeepOldNamestrue
\documentclass[letterpaper,10pt,english]{sphinxmanual}
\usepackage{iftex}

\ifPDFTeX
  \usepackage[utf8]{inputenc}
\fi
\ifdefined\DeclareUnicodeCharacter
  \DeclareUnicodeCharacter{00A0}{\nobreakspace}
\fi
\usepackage{cmap}
\usepackage[T1]{fontenc}
\usepackage{amsmath,amssymb,amstext}
\usepackage{babel}
\usepackage{times}
\usepackage[Bjarne]{fncychap}
\usepackage{longtable}
\usepackage{sphinx}
\usepackage{multirow}
\usepackage{eqparbox}


\addto\captionsenglish{\renewcommand{\figurename}{Fig.\@ }}
\addto\captionsenglish{\renewcommand{\tablename}{Table }}
\SetupFloatingEnvironment{literal-block}{name=Listing }

\addto\extrasenglish{\def\pageautorefname{page}}

\setcounter{tocdepth}{3}


\title{datamanager Documentation}
\date{Nov 18, 2016}
\release{1.0}
\author{William Digan Bastien Rance Hector Countouris}
\newcommand{\sphinxlogo}{}
\renewcommand{\releasename}{Release}
\makeindex

\makeatletter
\def\PYG@reset{\let\PYG@it=\relax \let\PYG@bf=\relax%
    \let\PYG@ul=\relax \let\PYG@tc=\relax%
    \let\PYG@bc=\relax \let\PYG@ff=\relax}
\def\PYG@tok#1{\csname PYG@tok@#1\endcsname}
\def\PYG@toks#1+{\ifx\relax#1\empty\else%
    \PYG@tok{#1}\expandafter\PYG@toks\fi}
\def\PYG@do#1{\PYG@bc{\PYG@tc{\PYG@ul{%
    \PYG@it{\PYG@bf{\PYG@ff{#1}}}}}}}
\def\PYG#1#2{\PYG@reset\PYG@toks#1+\relax+\PYG@do{#2}}

\expandafter\def\csname PYG@tok@gd\endcsname{\def\PYG@tc##1{\textcolor[rgb]{0.63,0.00,0.00}{##1}}}
\expandafter\def\csname PYG@tok@gu\endcsname{\let\PYG@bf=\textbf\def\PYG@tc##1{\textcolor[rgb]{0.50,0.00,0.50}{##1}}}
\expandafter\def\csname PYG@tok@gt\endcsname{\def\PYG@tc##1{\textcolor[rgb]{0.00,0.27,0.87}{##1}}}
\expandafter\def\csname PYG@tok@gs\endcsname{\let\PYG@bf=\textbf}
\expandafter\def\csname PYG@tok@gr\endcsname{\def\PYG@tc##1{\textcolor[rgb]{1.00,0.00,0.00}{##1}}}
\expandafter\def\csname PYG@tok@cm\endcsname{\let\PYG@it=\textit\def\PYG@tc##1{\textcolor[rgb]{0.25,0.50,0.56}{##1}}}
\expandafter\def\csname PYG@tok@vg\endcsname{\def\PYG@tc##1{\textcolor[rgb]{0.73,0.38,0.84}{##1}}}
\expandafter\def\csname PYG@tok@vi\endcsname{\def\PYG@tc##1{\textcolor[rgb]{0.73,0.38,0.84}{##1}}}
\expandafter\def\csname PYG@tok@mh\endcsname{\def\PYG@tc##1{\textcolor[rgb]{0.13,0.50,0.31}{##1}}}
\expandafter\def\csname PYG@tok@cs\endcsname{\def\PYG@tc##1{\textcolor[rgb]{0.25,0.50,0.56}{##1}}\def\PYG@bc##1{\setlength{\fboxsep}{0pt}\colorbox[rgb]{1.00,0.94,0.94}{\strut ##1}}}
\expandafter\def\csname PYG@tok@ge\endcsname{\let\PYG@it=\textit}
\expandafter\def\csname PYG@tok@vc\endcsname{\def\PYG@tc##1{\textcolor[rgb]{0.73,0.38,0.84}{##1}}}
\expandafter\def\csname PYG@tok@il\endcsname{\def\PYG@tc##1{\textcolor[rgb]{0.13,0.50,0.31}{##1}}}
\expandafter\def\csname PYG@tok@go\endcsname{\def\PYG@tc##1{\textcolor[rgb]{0.20,0.20,0.20}{##1}}}
\expandafter\def\csname PYG@tok@cp\endcsname{\def\PYG@tc##1{\textcolor[rgb]{0.00,0.44,0.13}{##1}}}
\expandafter\def\csname PYG@tok@gi\endcsname{\def\PYG@tc##1{\textcolor[rgb]{0.00,0.63,0.00}{##1}}}
\expandafter\def\csname PYG@tok@gh\endcsname{\let\PYG@bf=\textbf\def\PYG@tc##1{\textcolor[rgb]{0.00,0.00,0.50}{##1}}}
\expandafter\def\csname PYG@tok@ni\endcsname{\let\PYG@bf=\textbf\def\PYG@tc##1{\textcolor[rgb]{0.84,0.33,0.22}{##1}}}
\expandafter\def\csname PYG@tok@nl\endcsname{\let\PYG@bf=\textbf\def\PYG@tc##1{\textcolor[rgb]{0.00,0.13,0.44}{##1}}}
\expandafter\def\csname PYG@tok@nn\endcsname{\let\PYG@bf=\textbf\def\PYG@tc##1{\textcolor[rgb]{0.05,0.52,0.71}{##1}}}
\expandafter\def\csname PYG@tok@no\endcsname{\def\PYG@tc##1{\textcolor[rgb]{0.38,0.68,0.84}{##1}}}
\expandafter\def\csname PYG@tok@na\endcsname{\def\PYG@tc##1{\textcolor[rgb]{0.25,0.44,0.63}{##1}}}
\expandafter\def\csname PYG@tok@nb\endcsname{\def\PYG@tc##1{\textcolor[rgb]{0.00,0.44,0.13}{##1}}}
\expandafter\def\csname PYG@tok@nc\endcsname{\let\PYG@bf=\textbf\def\PYG@tc##1{\textcolor[rgb]{0.05,0.52,0.71}{##1}}}
\expandafter\def\csname PYG@tok@nd\endcsname{\let\PYG@bf=\textbf\def\PYG@tc##1{\textcolor[rgb]{0.33,0.33,0.33}{##1}}}
\expandafter\def\csname PYG@tok@ne\endcsname{\def\PYG@tc##1{\textcolor[rgb]{0.00,0.44,0.13}{##1}}}
\expandafter\def\csname PYG@tok@nf\endcsname{\def\PYG@tc##1{\textcolor[rgb]{0.02,0.16,0.49}{##1}}}
\expandafter\def\csname PYG@tok@si\endcsname{\let\PYG@it=\textit\def\PYG@tc##1{\textcolor[rgb]{0.44,0.63,0.82}{##1}}}
\expandafter\def\csname PYG@tok@s2\endcsname{\def\PYG@tc##1{\textcolor[rgb]{0.25,0.44,0.63}{##1}}}
\expandafter\def\csname PYG@tok@nt\endcsname{\let\PYG@bf=\textbf\def\PYG@tc##1{\textcolor[rgb]{0.02,0.16,0.45}{##1}}}
\expandafter\def\csname PYG@tok@nv\endcsname{\def\PYG@tc##1{\textcolor[rgb]{0.73,0.38,0.84}{##1}}}
\expandafter\def\csname PYG@tok@s1\endcsname{\def\PYG@tc##1{\textcolor[rgb]{0.25,0.44,0.63}{##1}}}
\expandafter\def\csname PYG@tok@ch\endcsname{\let\PYG@it=\textit\def\PYG@tc##1{\textcolor[rgb]{0.25,0.50,0.56}{##1}}}
\expandafter\def\csname PYG@tok@m\endcsname{\def\PYG@tc##1{\textcolor[rgb]{0.13,0.50,0.31}{##1}}}
\expandafter\def\csname PYG@tok@gp\endcsname{\let\PYG@bf=\textbf\def\PYG@tc##1{\textcolor[rgb]{0.78,0.36,0.04}{##1}}}
\expandafter\def\csname PYG@tok@sh\endcsname{\def\PYG@tc##1{\textcolor[rgb]{0.25,0.44,0.63}{##1}}}
\expandafter\def\csname PYG@tok@ow\endcsname{\let\PYG@bf=\textbf\def\PYG@tc##1{\textcolor[rgb]{0.00,0.44,0.13}{##1}}}
\expandafter\def\csname PYG@tok@sx\endcsname{\def\PYG@tc##1{\textcolor[rgb]{0.78,0.36,0.04}{##1}}}
\expandafter\def\csname PYG@tok@bp\endcsname{\def\PYG@tc##1{\textcolor[rgb]{0.00,0.44,0.13}{##1}}}
\expandafter\def\csname PYG@tok@c1\endcsname{\let\PYG@it=\textit\def\PYG@tc##1{\textcolor[rgb]{0.25,0.50,0.56}{##1}}}
\expandafter\def\csname PYG@tok@o\endcsname{\def\PYG@tc##1{\textcolor[rgb]{0.40,0.40,0.40}{##1}}}
\expandafter\def\csname PYG@tok@kc\endcsname{\let\PYG@bf=\textbf\def\PYG@tc##1{\textcolor[rgb]{0.00,0.44,0.13}{##1}}}
\expandafter\def\csname PYG@tok@c\endcsname{\let\PYG@it=\textit\def\PYG@tc##1{\textcolor[rgb]{0.25,0.50,0.56}{##1}}}
\expandafter\def\csname PYG@tok@mf\endcsname{\def\PYG@tc##1{\textcolor[rgb]{0.13,0.50,0.31}{##1}}}
\expandafter\def\csname PYG@tok@err\endcsname{\def\PYG@bc##1{\setlength{\fboxsep}{0pt}\fcolorbox[rgb]{1.00,0.00,0.00}{1,1,1}{\strut ##1}}}
\expandafter\def\csname PYG@tok@mb\endcsname{\def\PYG@tc##1{\textcolor[rgb]{0.13,0.50,0.31}{##1}}}
\expandafter\def\csname PYG@tok@ss\endcsname{\def\PYG@tc##1{\textcolor[rgb]{0.32,0.47,0.09}{##1}}}
\expandafter\def\csname PYG@tok@sr\endcsname{\def\PYG@tc##1{\textcolor[rgb]{0.14,0.33,0.53}{##1}}}
\expandafter\def\csname PYG@tok@mo\endcsname{\def\PYG@tc##1{\textcolor[rgb]{0.13,0.50,0.31}{##1}}}
\expandafter\def\csname PYG@tok@kd\endcsname{\let\PYG@bf=\textbf\def\PYG@tc##1{\textcolor[rgb]{0.00,0.44,0.13}{##1}}}
\expandafter\def\csname PYG@tok@mi\endcsname{\def\PYG@tc##1{\textcolor[rgb]{0.13,0.50,0.31}{##1}}}
\expandafter\def\csname PYG@tok@kn\endcsname{\let\PYG@bf=\textbf\def\PYG@tc##1{\textcolor[rgb]{0.00,0.44,0.13}{##1}}}
\expandafter\def\csname PYG@tok@cpf\endcsname{\let\PYG@it=\textit\def\PYG@tc##1{\textcolor[rgb]{0.25,0.50,0.56}{##1}}}
\expandafter\def\csname PYG@tok@kr\endcsname{\let\PYG@bf=\textbf\def\PYG@tc##1{\textcolor[rgb]{0.00,0.44,0.13}{##1}}}
\expandafter\def\csname PYG@tok@s\endcsname{\def\PYG@tc##1{\textcolor[rgb]{0.25,0.44,0.63}{##1}}}
\expandafter\def\csname PYG@tok@kp\endcsname{\def\PYG@tc##1{\textcolor[rgb]{0.00,0.44,0.13}{##1}}}
\expandafter\def\csname PYG@tok@w\endcsname{\def\PYG@tc##1{\textcolor[rgb]{0.73,0.73,0.73}{##1}}}
\expandafter\def\csname PYG@tok@kt\endcsname{\def\PYG@tc##1{\textcolor[rgb]{0.56,0.13,0.00}{##1}}}
\expandafter\def\csname PYG@tok@sc\endcsname{\def\PYG@tc##1{\textcolor[rgb]{0.25,0.44,0.63}{##1}}}
\expandafter\def\csname PYG@tok@sb\endcsname{\def\PYG@tc##1{\textcolor[rgb]{0.25,0.44,0.63}{##1}}}
\expandafter\def\csname PYG@tok@k\endcsname{\let\PYG@bf=\textbf\def\PYG@tc##1{\textcolor[rgb]{0.00,0.44,0.13}{##1}}}
\expandafter\def\csname PYG@tok@se\endcsname{\let\PYG@bf=\textbf\def\PYG@tc##1{\textcolor[rgb]{0.25,0.44,0.63}{##1}}}
\expandafter\def\csname PYG@tok@sd\endcsname{\let\PYG@it=\textit\def\PYG@tc##1{\textcolor[rgb]{0.25,0.44,0.63}{##1}}}

\def\PYGZbs{\char`\\}
\def\PYGZus{\char`\_}
\def\PYGZob{\char`\{}
\def\PYGZcb{\char`\}}
\def\PYGZca{\char`\^}
\def\PYGZam{\char`\&}
\def\PYGZlt{\char`\<}
\def\PYGZgt{\char`\>}
\def\PYGZsh{\char`\#}
\def\PYGZpc{\char`\%}
\def\PYGZdl{\char`\$}
\def\PYGZhy{\char`\-}
\def\PYGZsq{\char`\'}
\def\PYGZdq{\char`\"}
\def\PYGZti{\char`\~}
% for compatibility with earlier versions
\def\PYGZat{@}
\def\PYGZlb{[}
\def\PYGZrb{]}
\makeatother

\renewcommand\PYGZsq{\textquotesingle}

\begin{document}

\maketitle
\tableofcontents
\phantomsection\label{index::doc}


Contents:


\chapter{datamanagerpkg Package}
\label{datamanagerpkg:datamanagerpkg-package}\label{datamanagerpkg:welcome-to-datamanager-s-documentation}\label{datamanagerpkg::doc}

\section{\texttt{datamanagerpkg} Package}
\label{datamanagerpkg:id1}\phantomsection\label{datamanagerpkg:module-datamanagerpkg.__init__}\index{datamanagerpkg.\_\_init\_\_ (module)}\index{grou() (in module datamanagerpkg.\_\_init\_\_)}

\begin{fulllineitems}
\phantomsection\label{datamanagerpkg:datamanagerpkg.__init__.grou}\pysiglinewithargsret{\sphinxcode{datamanagerpkg.\_\_init\_\_.}\sphinxbfcode{grou}}{}{}
\end{fulllineitems}



\section{\texttt{GalaxyCommunication\_data\_manager} Module}
\label{datamanagerpkg:galaxycommunication-data-manager-module}\label{datamanagerpkg:module-datamanagerpkg.GalaxyCommunication_data_manager}\index{datamanagerpkg.GalaxyCommunication\_data\_manager (module)}
This module illustrates how to write 
GalaxyCommunication\_data\_manager.pyc
and ProtonCommunication\_data\_manager.py
Basically it is just a sphinx test for the documentation
\index{CNV\_Input\_Dict() (in module datamanagerpkg.GalaxyCommunication\_data\_manager)}

\begin{fulllineitems}
\phantomsection\label{datamanagerpkg:datamanagerpkg.GalaxyCommunication_data_manager.CNV_Input_Dict}\pysiglinewithargsret{\sphinxcode{datamanagerpkg.GalaxyCommunication\_data\_manager.}\sphinxbfcode{CNV\_Input\_Dict}}{\emph{galaxyWeb}, \emph{historyID}}{}
returns (data\_Input\_CNVID)

\textbf{Descriptions}:

This function return a dictionnary whitch contains datasets id for 
CNV input files. This dictionnary contains a bcsummary and bcmatrix keys.

\textbf{Parameters}:
\begin{quote}\begin{description}
\item[{Parameters}] \leavevmode\begin{itemize}
\item {} 
\textbf{\texttt{galaxyWeb}} (\emph{\texttt{GalaxyInstance}}) -- a connection to your galaxy instance

\item {} 
\textbf{\texttt{historyID}} (\emph{\texttt{string}}) -- a galaxy history ID

\end{itemize}

\item[{Returns}] \leavevmode
data\_Input\_CNVID

\item[{Return type}] \leavevmode
dictionnary

\end{description}\end{quote}

\end{fulllineitems}

\index{Create\_History() (in module datamanagerpkg.GalaxyCommunication\_data\_manager)}

\begin{fulllineitems}
\phantomsection\label{datamanagerpkg:datamanagerpkg.GalaxyCommunication_data_manager.Create_History}\pysiglinewithargsret{\sphinxcode{datamanagerpkg.GalaxyCommunication\_data\_manager.}\sphinxbfcode{Create\_History}}{\emph{galaxyWeb}, \emph{workflow\_Name}}{}
returns (historyDict)

\textbf{Descriptions}:

This function create a new galaxy history where the data will be load.

\textbf{Parameters}:
\begin{quote}\begin{description}
\item[{Parameters}] \leavevmode\begin{itemize}
\item {} 
\textbf{\texttt{galaxyWeb}} (\emph{\texttt{GalaxyInstance}}) -- a connection to your galaxy instance

\item {} 
\textbf{\texttt{workflow\_Name}} (\emph{\texttt{string}}) -- part of the name of the history

\end{itemize}

\item[{Returns}] \leavevmode
historyDict

\item[{Return type}] \leavevmode
dict

\end{description}\end{quote}

\end{fulllineitems}

\index{Run\_CNV\_Workflow() (in module datamanagerpkg.GalaxyCommunication\_data\_manager)}

\begin{fulllineitems}
\phantomsection\label{datamanagerpkg:datamanagerpkg.GalaxyCommunication_data_manager.Run_CNV_Workflow}\pysiglinewithargsret{\sphinxcode{datamanagerpkg.GalaxyCommunication\_data\_manager.}\sphinxbfcode{Run\_CNV\_Workflow}}{\emph{galaxyWeb}, \emph{data\_Input\_CNVID}, \emph{historyID}}{}
returns (int)

\textbf{Descriptions}:

This function retrieve the CNV workflow and execute it. Use a dictionnary
as input.

\textbf{Parameters}:
\begin{quote}\begin{description}
\item[{Parameters}] \leavevmode\begin{itemize}
\item {} 
\textbf{\texttt{galaxyWeb}} (\emph{\texttt{GalaxyInstance}}) -- a connection to your galaxy instance

\item {} 
\textbf{\texttt{data\_Input\_CNVID}} (\emph{\texttt{dictionnary}}) -- a dictionnary output from function CNV\_Input\_Dict

\item {} 
\textbf{\texttt{historyID}} (\emph{\texttt{string}}) -- a galaxy history ID

\end{itemize}

\item[{Returns}] \leavevmode
1

\item[{Return type}] \leavevmode
int

\end{description}\end{quote}

\end{fulllineitems}

\index{addAllWorkflow() (in module datamanagerpkg.GalaxyCommunication\_data\_manager)}

\begin{fulllineitems}
\phantomsection\label{datamanagerpkg:datamanagerpkg.GalaxyCommunication_data_manager.addAllWorkflow}\pysiglinewithargsret{\sphinxcode{datamanagerpkg.GalaxyCommunication\_data\_manager.}\sphinxbfcode{addAllWorkflow}}{\emph{galaxyWeb}, \emph{workflow\_Dir}}{}
returns (int)

\textbf{Descriptions}:

This function aims to load all workflows on a folder such as 
`/nas\_Dir/workflow' for the current users.

\textbf{Parameters}:
\begin{quote}\begin{description}
\item[{Parameters}] \leavevmode\begin{itemize}
\item {} 
\textbf{\texttt{galaxyWeb}} (\emph{\texttt{GalaxyInstance}}) -- a connection to your galaxy instance

\item {} 
\textbf{\texttt{workflow\_Dir}} (\emph{\texttt{string}}) -- path to the workflow directory

\end{itemize}

\item[{Returns}] \leavevmode
0 or 1

\item[{Return type}] \leavevmode
int

\end{description}\end{quote}

\begin{notice}{note}{Note:}
This function need to be used only one time when the
\end{notice}

Galaxy user api key is generated

\end{fulllineitems}

\index{createUserApikey() (in module datamanagerpkg.GalaxyCommunication\_data\_manager)}

\begin{fulllineitems}
\phantomsection\label{datamanagerpkg:datamanagerpkg.GalaxyCommunication_data_manager.createUserApikey}\pysiglinewithargsret{\sphinxcode{datamanagerpkg.GalaxyCommunication\_data\_manager.}\sphinxbfcode{createUserApikey}}{\emph{galaxyWeb}, \emph{userID}}{}
returns (userApiKey)

\textbf{Descriptions}:

This function aims to return the galaxy users dictionnary.

\textbf{Parameters}:
\begin{quote}\begin{description}
\item[{Parameters}] \leavevmode\begin{itemize}
\item {} 
\textbf{\texttt{galaxyWeb}} (\emph{\texttt{GalaxyInstance}}) -- a connection to your galaxy instance

\item {} 
\textbf{\texttt{userID}} (\emph{\texttt{string}}) -- the current user ID in  Galaxy

\end{itemize}

\item[{Returns}] \leavevmode
userApiKey

\item[{Return type}] \leavevmode
string

\end{description}\end{quote}

\begin{notice}{note}{Note:}
In this function I can not use the users.get\_current\_user()
function from bioblend because I use the Galaxy Master ApiKey
\end{notice}

\end{fulllineitems}

\index{galaxyConnection() (in module datamanagerpkg.GalaxyCommunication\_data\_manager)}

\begin{fulllineitems}
\phantomsection\label{datamanagerpkg:datamanagerpkg.GalaxyCommunication_data_manager.galaxyConnection}\pysiglinewithargsret{\sphinxcode{datamanagerpkg.GalaxyCommunication\_data\_manager.}\sphinxbfcode{galaxyConnection}}{\emph{base\_url}, \emph{apiKey}}{}
returns (GalaxyInstance)

\textbf{Descriptions}:

This function aims to create a connection to the Galaxy server.

\textbf{Parameters}:
\begin{quote}\begin{description}
\item[{Parameters}] \leavevmode\begin{itemize}
\item {} 
\textbf{\texttt{base\_url}} (\emph{\texttt{string}}) -- an url which point to your galaxy instance

\item {} 
\textbf{\texttt{apiKey}} (\emph{\texttt{string}}) -- a valid galaxy API key

\end{itemize}

\item[{Returns}] \leavevmode
GalaxyInstance

\item[{Return type}] \leavevmode
GalaxyInstance

\end{description}\end{quote}

\end{fulllineitems}

\index{mainCNV() (in module datamanagerpkg.GalaxyCommunication\_data\_manager)}

\begin{fulllineitems}
\phantomsection\label{datamanagerpkg:datamanagerpkg.GalaxyCommunication_data_manager.mainCNV}\pysiglinewithargsret{\sphinxcode{datamanagerpkg.GalaxyCommunication\_data\_manager.}\sphinxbfcode{mainCNV}}{\emph{expDict}, \emph{base\_url}, \emph{apiKey}}{}
returns (historyID)

\textbf{Descriptions}:

This function execute the CNV routine. From a run of the Ion Proton,
The routine will connect the user to Galaxy, create an history,
upload the CNV input files to it and run the CNV workflow.

\textbf{Parameters}:
\begin{quote}\begin{description}
\item[{Parameters}] \leavevmode\begin{itemize}
\item {} 
\textbf{\texttt{expDict}} -- a dictionnary output from ProtonCommunication\_data\_manager.copyData().

\item {} 
\textbf{\texttt{base\_url}} (\emph{\texttt{string}}) -- an url which point to your galaxy instance

\item {} 
\textbf{\texttt{apiKey}} (\emph{\texttt{string}}) -- a valid galaxy API key

\end{itemize}

\item[{Returns historyID}] \leavevmode
the galaxy history where the data and the CNV run are located

\item[{Rtype historyID}] \leavevmode
a dictionnary

\end{description}\end{quote}

\end{fulllineitems}

\index{returnGalaxyUsers() (in module datamanagerpkg.GalaxyCommunication\_data\_manager)}

\begin{fulllineitems}
\phantomsection\label{datamanagerpkg:datamanagerpkg.GalaxyCommunication_data_manager.returnGalaxyUsers}\pysiglinewithargsret{\sphinxcode{datamanagerpkg.GalaxyCommunication\_data\_manager.}\sphinxbfcode{returnGalaxyUsers}}{\emph{galaxyWeb}}{}
returns (usersDict)

\textbf{Descriptions}:

This function aims to return the galaxy users dictionnary.

\textbf{Parameters}:
\begin{quote}\begin{description}
\item[{Parameters}] \leavevmode
\textbf{\texttt{galaxyWeb}} (\emph{\texttt{GalaxyInstance}}) -- a connection to your galaxy instance

\item[{Returns}] \leavevmode
usersDict

\item[{Return type}] \leavevmode
Dictionnary

\end{description}\end{quote}

\begin{notice}{note}{Note:}
In this function I can not use the users.get\_current\_user()
function from bioblend because I use the Galaxy Master ApiKey
\end{notice}

\end{fulllineitems}

\index{upload\_To\_History\_CNV() (in module datamanagerpkg.GalaxyCommunication\_data\_manager)}

\begin{fulllineitems}
\phantomsection\label{datamanagerpkg:datamanagerpkg.GalaxyCommunication_data_manager.upload_To_History_CNV}\pysiglinewithargsret{\sphinxcode{datamanagerpkg.GalaxyCommunication\_data\_manager.}\sphinxbfcode{upload\_To\_History\_CNV}}{\emph{galaxyWeb}, \emph{expDict}, \emph{historyID}}{}
returns (int)

\textbf{Descriptions}:

This function upload to a specific history the CNV data.

\textbf{Parameters}:
\begin{quote}\begin{description}
\item[{Parameters}] \leavevmode\begin{itemize}
\item {} 
\textbf{\texttt{galaxyWeb}} (\emph{\texttt{GalaxyInstance}}) -- a connection to your galaxy instance

\item {} 
\textbf{\texttt{expDict}} (\emph{\texttt{dictionnary}}) -- a result dictionnary output from the ProtonCommunication script

\item {} 
\textbf{\texttt{historyID}} (\emph{\texttt{string}}) -- a galaxy history ID

\end{itemize}

\item[{Returns}] \leavevmode
1

\item[{Return type}] \leavevmode
int

\end{description}\end{quote}

\end{fulllineitems}



\section{\texttt{Main\_data\_manager} Module}
\label{datamanagerpkg:module-datamanagerpkg.Main_data_manager}\label{datamanagerpkg:main-data-manager-module}\index{datamanagerpkg.Main\_data\_manager (module)}

\section{\texttt{ProtonCommunication\_data\_manager} Module}
\label{datamanagerpkg:protoncommunication-data-manager-module}\label{datamanagerpkg:module-datamanagerpkg.ProtonCommunication_data_manager}\index{datamanagerpkg.ProtonCommunication\_data\_manager (module)}
The module ProtonCommunication\_data\_manager.py was
designed to be able to connect to the HEGP Ion Proton and
copy Data easily. It can be used with GalaxyCommunication\_data\_manager.py which
assure the Data-Manager\_Galaxy Job routine.

This script use the The Torrent Suite Software Development Kit
to communicate with the Ion Proton.
ProtonCommunication\_data\_manager fullfill three main goals:
- retrieve the Data
- Select the data you want to use
- Copy them throught the network
\index{CheckExperiments() (in module datamanagerpkg.ProtonCommunication\_data\_manager)}

\begin{fulllineitems}
\phantomsection\label{datamanagerpkg:datamanagerpkg.ProtonCommunication_data_manager.CheckExperiments}\pysiglinewithargsret{\sphinxcode{datamanagerpkg.ProtonCommunication\_data\_manager.}\sphinxbfcode{CheckExperiments}}{\emph{nb\_limit}, \emph{idpwd}, \emph{base\_url}}{}
returns (dictionnary)

\textbf{Descriptions}:
\begin{description}
\item[{This function aims to return a dictionnary, which contains the `n'}] \leavevmode
last experiments. It also return the run status and if the run is Complete or not.
For that purpose you need to provide the following parameters.

\end{description}

\textbf{Parameters}:
\begin{quote}\begin{description}
\item[{Parameters}] \leavevmode\begin{itemize}
\item {} 
\textbf{\texttt{nb\_limit}} (\emph{\texttt{int}}) -- the `n' number of experiments to check out

\item {} 
\textbf{\texttt{idpwd}} (\emph{\texttt{string}}) -- the user ID to connect to the proton

\item {} 
\textbf{\texttt{idpwd}} -- the user ID to connect to the proton(To add)

\item {} 
\textbf{\texttt{base\_url}} (\emph{\texttt{string}}) -- the Ion Proton URL

\end{itemize}

\item[{Returns}] \leavevmode
dict

\item[{Return type}] \leavevmode

dict
\begin{quote}
\begin{quote}

This function retrieve the n last experiment of the Ion Proton.
it returns a dictionnary which contains 5 elements.
\{RunName: \{cnvFileName;status;ftpStatus;date;id;resultsQuery\}\}
\end{quote}

\begin{notice}{note}{Note:}
Dictionnary structure: \{RunName: \{cnvFileName;status;ftpStatus;date;id;resultsQuery\}\}
\end{notice}
\end{quote}
\begin{itemize}
\item {} 
RunName : the experiments run name

\item {} 
cnvFileName : the sting match for the bcsummary and bcmatrix file

\item {} 
status :  run status either `pending' or `run'

\item {} 
ftpStatus : if the data can be download, either `Complete' or `'

\item {} 
date : project date

\item {} 
id : project id in the ion proton

\item {} 
resultsQuery: astring to the result folder {}`

\end{itemize}


\end{description}\end{quote}

\end{fulllineitems}

\index{CheckResConsistency() (in module datamanagerpkg.ProtonCommunication\_data\_manager)}

\begin{fulllineitems}
\phantomsection\label{datamanagerpkg:datamanagerpkg.ProtonCommunication_data_manager.CheckResConsistency}\pysiglinewithargsret{\sphinxcode{datamanagerpkg.ProtonCommunication\_data\_manager.}\sphinxbfcode{CheckResConsistency}}{\emph{expDict}, \emph{ssh}}{}
returns (dictionnary)

\textbf{Descriptions}:

This function  check data consistency for one dictionnary before performed a scp command.
Quality control need to be handle in this function. add a key coverageAnalysis\_out
which point to the right folder coverageAnalysis\_out which contains the bed file ColonLungV2.20140523

\textbf{Parameters}:
\begin{quote}\begin{description}
\item[{Parameters}] \leavevmode\begin{itemize}
\item {} 
\textbf{\texttt{expDict}} -- a dictionnary output from QueryResults()

\item {} 
\textbf{\texttt{ssh}} ({\hyperref[datamanagerpkg:datamanagerpkg.ProtonCommunication_data_manager.sshConnection]{\sphinxcrossref{\emph{\texttt{sshConnection}}}}}) -- sshConnection from sshConnection()
:type expDict: dict

\end{itemize}

\item[{Returns}] \leavevmode
dict

\item[{Return type}] \leavevmode
dict

\end{description}\end{quote}

\end{fulllineitems}

\index{CheckResultsConsistency() (in module datamanagerpkg.ProtonCommunication\_data\_manager)}

\begin{fulllineitems}
\phantomsection\label{datamanagerpkg:datamanagerpkg.ProtonCommunication_data_manager.CheckResultsConsistency}\pysiglinewithargsret{\sphinxcode{datamanagerpkg.ProtonCommunication\_data\_manager.}\sphinxbfcode{CheckResultsConsistency}}{\emph{expDict}, \emph{ssh}}{}
returns (dictionnary)

\textbf{Descriptions}:

This function  check data consistency for a a dict of dictionnary before performed a scp command.
Quality control need to be handle in this function. add a key coverageAnalysis\_out
which point to the right folder coverageAnalysis\_out which contains the bed file ColonLungV2.20140523

\textbf{Parameters}:
\begin{quote}\begin{description}
\item[{Parameters}] \leavevmode\begin{itemize}
\item {} 
\textbf{\texttt{expDict}} -- a dictionnary output from QueryResults()

\item {} 
\textbf{\texttt{ssh}} ({\hyperref[datamanagerpkg:datamanagerpkg.ProtonCommunication_data_manager.sshConnection]{\sphinxcrossref{\emph{\texttt{sshConnection}}}}}) -- sshConnection from sshConnection()
:type expDict: dict

\end{itemize}

\item[{Returns}] \leavevmode
dict

\item[{Return type}] \leavevmode
dict

\end{description}\end{quote}

\end{fulllineitems}

\index{FindResults() (in module datamanagerpkg.ProtonCommunication\_data\_manager)}

\begin{fulllineitems}
\phantomsection\label{datamanagerpkg:datamanagerpkg.ProtonCommunication_data_manager.FindResults}\pysiglinewithargsret{\sphinxcode{datamanagerpkg.ProtonCommunication\_data\_manager.}\sphinxbfcode{FindResults}}{\emph{expDict}, \emph{idpwd}, \emph{base\_url}}{}
returns (dictionnary)

\textbf{Descriptions}:

This function find the result folder absolut path. it retuns a dictionnary 
and add the filesystempath to the current dictionnary.

\textbf{Parameters}:
\begin{quote}\begin{description}
\item[{Parameters}] \leavevmode\begin{itemize}
\item {} 
\textbf{\texttt{expDict}} (\emph{\texttt{dict}}) -- a directory output from the the CheckExperiments() function

\item {} 
\textbf{\texttt{idpwd}} (\emph{\texttt{string}}) -- the user ID to connect to the proton

\item {} 
\textbf{\texttt{idpwd}} -- the user ID to connect to the proton(To add)

\item {} 
\textbf{\texttt{base\_url}} (\emph{\texttt{string}}) -- the Ion Proton URL

\end{itemize}

\item[{Returns}] \leavevmode
dict

\item[{Return type}] \leavevmode

dict
\begin{quote}
\begin{quote}
\begin{description}
\item[{This function retrieve the result path associated with an}] \leavevmode
experiments name from the Ion Proton.

\item[{it returns a dictionnary which contains 2 new elements}] \leavevmode
from the current dictionnary.

\end{description}

\{RunName: \{resultsName;runPath;...\}\}
\end{quote}

\begin{notice}{note}{Note:}
Dictionnary structure: \{RunName: \{resultsName;runPath;cnvFileName;status;ftpStatus;date;id;resultsQuery\}\}
\end{notice}
\end{quote}
\begin{itemize}
\item {} 
resultsName; : the experiments result name

\item {} 
runPath : the path to the current result folder in the Ion Proton

\end{itemize}


\end{description}\end{quote}

\end{fulllineitems}

\index{QueryResults() (in module datamanagerpkg.ProtonCommunication\_data\_manager)}

\begin{fulllineitems}
\phantomsection\label{datamanagerpkg:datamanagerpkg.ProtonCommunication_data_manager.QueryResults}\pysiglinewithargsret{\sphinxcode{datamanagerpkg.ProtonCommunication\_data\_manager.}\sphinxbfcode{QueryResults}}{\emph{resultsQuery}, \emph{idpwd}, \emph{base\_url}}{}
returns (dictionnary)

\textbf{Descriptions}:

This function find the result folder absolut path associated to an
identified resultsQuery. it returns a dictionnary 
and add the filesystempath to the current dictionnary.

\textbf{Parameters}:
\begin{quote}\begin{description}
\item[{Parameters}] \leavevmode\begin{itemize}
\item {} 
\textbf{\texttt{resultsQuery}} -- a string resultsQuery output from CheckExperiments()

\item {} 
\textbf{\texttt{idpwd}} (\emph{\texttt{string}}) -- the user ID to connect to the proton

\item {} 
\textbf{\texttt{idpwd}} -- the user ID to connect to the proton(To add)

\item {} 
\textbf{\texttt{base\_url}} (\emph{\texttt{string}}) -- the Ion Proton URL

\end{itemize}

\item[{Returns}] \leavevmode
dict

\item[{Return type}] \leavevmode

dict
\begin{quote}
\begin{quote}
\begin{description}
\item[{This function retrieve the result path associated with an}] \leavevmode
experiments name from the Ion Proton.

\item[{it returns a dictionnary which contains 2 new elements}] \leavevmode
from the current dictionnary.

\end{description}

\{RunName: \{cnvFileName;resultsName;runPath;\}\}
\end{quote}

\begin{notice}{note}{Note:}
Dictionnary structure: \{RunName: \{resultsName;runPath;cnvFileName\}\}
\end{notice}
\end{quote}
\begin{itemize}
\item {} 
resultsName; : the experiments result name

\item {} 
runPath : the path to the current result folder in the Ion Proton

\end{itemize}


\end{description}\end{quote}

\end{fulllineitems}

\index{copyData() (in module datamanagerpkg.ProtonCommunication\_data\_manager)}

\begin{fulllineitems}
\phantomsection\label{datamanagerpkg:datamanagerpkg.ProtonCommunication_data_manager.copyData}\pysiglinewithargsret{\sphinxcode{datamanagerpkg.ProtonCommunication\_data\_manager.}\sphinxbfcode{copyData}}{\emph{currentExp}, \emph{ssh}}{}
returns (dictionnary)

\textbf{Descriptions}:

This function copy data trought scp and perform checksum. Add the key bcmatrix
and bcsummary to the current directory. if some rename opperation need to be performed
it as to been done here.

\textbf{Parameters}:
\begin{quote}\begin{description}
\item[{Parameters}] \leavevmode\begin{itemize}
\item {} 
\textbf{\texttt{currentExp}} (\emph{\texttt{dict}}) -- a directory output from CheckResultsConsistency()

\item {} 
\textbf{\texttt{ssh}} ({\hyperref[datamanagerpkg:datamanagerpkg.ProtonCommunication_data_manager.sshConnection]{\sphinxcrossref{\emph{\texttt{sshConnection}}}}}) -- sshConnection from sshConnection()

\end{itemize}

\item[{Returns}] \leavevmode
dict

\item[{Return type}] \leavevmode
dict

\end{description}\end{quote}

\end{fulllineitems}

\index{mainGetCNVData() (in module datamanagerpkg.ProtonCommunication\_data\_manager)}

\begin{fulllineitems}
\phantomsection\label{datamanagerpkg:datamanagerpkg.ProtonCommunication_data_manager.mainGetCNVData}\pysiglinewithargsret{\sphinxcode{datamanagerpkg.ProtonCommunication\_data\_manager.}\sphinxbfcode{mainGetCNVData}}{\emph{base\_url}, \emph{idpr}, \emph{severName}, \emph{experimentLimit}}{}
\end{fulllineitems}

\index{sshConnection() (in module datamanagerpkg.ProtonCommunication\_data\_manager)}

\begin{fulllineitems}
\phantomsection\label{datamanagerpkg:datamanagerpkg.ProtonCommunication_data_manager.sshConnection}\pysiglinewithargsret{\sphinxcode{datamanagerpkg.ProtonCommunication\_data\_manager.}\sphinxbfcode{sshConnection}}{\emph{severName}, \emph{idpwd}}{}
returns (sshConnection)

\textbf{Descriptions}:

This function allow an ssh connection through the  pakito python module.
the goal here is to establish a connection before performed an scp bash
command.

\textbf{Parameters}:
:param severName: name of the linux machine to connect throught ssh
:param idpwd: the user ID to connect to the proton
:param idpwd: the user ID to connect to the proton(To add) 
:type severName: string
:type idpwd: string
:type idpwd: string
:returns: sshConnection
:rtype: sshConnection

\end{fulllineitems}



\section{\texttt{addWorkflow} Module}
\label{datamanagerpkg:module-datamanagerpkg.addWorkflow}\label{datamanagerpkg:addworkflow-module}\index{datamanagerpkg.addWorkflow (module)}\index{CNV\_Input\_Dict() (in module datamanagerpkg.addWorkflow)}

\begin{fulllineitems}
\phantomsection\label{datamanagerpkg:datamanagerpkg.addWorkflow.CNV_Input_Dict}\pysiglinewithargsret{\sphinxcode{datamanagerpkg.addWorkflow.}\sphinxbfcode{CNV\_Input\_Dict}}{\emph{galaxyWeb}, \emph{historyID}}{}
\end{fulllineitems}

\index{Create\_History() (in module datamanagerpkg.addWorkflow)}

\begin{fulllineitems}
\phantomsection\label{datamanagerpkg:datamanagerpkg.addWorkflow.Create_History}\pysiglinewithargsret{\sphinxcode{datamanagerpkg.addWorkflow.}\sphinxbfcode{Create\_History}}{\emph{galaxyWeb}, \emph{workflow\_Name}}{}
\end{fulllineitems}

\index{Run\_CNV\_Workflow() (in module datamanagerpkg.addWorkflow)}

\begin{fulllineitems}
\phantomsection\label{datamanagerpkg:datamanagerpkg.addWorkflow.Run_CNV_Workflow}\pysiglinewithargsret{\sphinxcode{datamanagerpkg.addWorkflow.}\sphinxbfcode{Run\_CNV\_Workflow}}{\emph{galaxyWeb}, \emph{data\_Input\_CNVID}, \emph{historyID}}{}
\end{fulllineitems}

\index{addAllWorkflow() (in module datamanagerpkg.addWorkflow)}

\begin{fulllineitems}
\phantomsection\label{datamanagerpkg:datamanagerpkg.addWorkflow.addAllWorkflow}\pysiglinewithargsret{\sphinxcode{datamanagerpkg.addWorkflow.}\sphinxbfcode{addAllWorkflow}}{\emph{galaxyWeb}, \emph{workflow\_Dir}}{}
\end{fulllineitems}

\index{mainCNV() (in module datamanagerpkg.addWorkflow)}

\begin{fulllineitems}
\phantomsection\label{datamanagerpkg:datamanagerpkg.addWorkflow.mainCNV}\pysiglinewithargsret{\sphinxcode{datamanagerpkg.addWorkflow.}\sphinxbfcode{mainCNV}}{\emph{pathToFile}, \emph{apiKey}, \emph{inputAbsolutPath}}{}
\end{fulllineitems}

\index{upload\_To\_History() (in module datamanagerpkg.addWorkflow)}

\begin{fulllineitems}
\phantomsection\label{datamanagerpkg:datamanagerpkg.addWorkflow.upload_To_History}\pysiglinewithargsret{\sphinxcode{datamanagerpkg.addWorkflow.}\sphinxbfcode{upload\_To\_History}}{\emph{galaxyWeb}, \emph{filesPath}, \emph{historyID}, \emph{inputAbsolutPath}}{}
\end{fulllineitems}



\chapter{Indices and tables}
\label{index:indices-and-tables}\begin{itemize}
\item {} 
\DUrole{xref,std,std-ref}{genindex}

\item {} 
\DUrole{xref,std,std-ref}{modindex}

\item {} 
\DUrole{xref,std,std-ref}{search}

\end{itemize}


\renewcommand{\indexname}{Python Module Index}
\begin{theindex}
\def\bigletter#1{{\Large\sffamily#1}\nopagebreak\vspace{1mm}}
\bigletter{d}
\item {\texttt{datamanagerpkg.\_\_init\_\_}}, \pageref{datamanagerpkg:module-datamanagerpkg.__init__}
\item {\texttt{datamanagerpkg.addWorkflow}}, \pageref{datamanagerpkg:module-datamanagerpkg.addWorkflow}
\item {\texttt{datamanagerpkg.GalaxyCommunication\_data\_manager}}, \pageref{datamanagerpkg:module-datamanagerpkg.GalaxyCommunication_data_manager}
\item {\texttt{datamanagerpkg.Main\_data\_manager}}, \pageref{datamanagerpkg:module-datamanagerpkg.Main_data_manager}
\item {\texttt{datamanagerpkg.ProtonCommunication\_data\_manager}}, \pageref{datamanagerpkg:module-datamanagerpkg.ProtonCommunication_data_manager}
\end{theindex}

\renewcommand{\indexname}{Index}
\printindex
\end{document}
